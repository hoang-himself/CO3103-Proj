\documentclass[a4paper]{article}

% Global layout
\usepackage{fancyhdr, graphicx, hyperref, indentfirst, lastpage, setspace}
\usepackage{geometry}

% Encoding
\usepackage[utf8]{vntex, inputenc}
\usepackage[english]{babel}
\usepackage{amsmath, amssymb, gensymb}

% Better table
\usepackage{array, booktabs, multicol, multirow, siunitx, tabularx}

% Code space
\usepackage[dvipsnames]{xcolor}
\usepackage{tikz}
\usepackage[framemethod=tikz]{mdframed}
\usepackage{minted, verbatim} % needs --shell-escape flag and Pygments

% Graphics
\usepackage{caption, float}

% Bullets & numbering
\usepackage{enumitem}

% Page setup
\allowdisplaybreaks{} % to have page breaks inside align* environment
\hypersetup{urlcolor=blue,linkcolor=black,citecolor=red,colorlinks=true}
\usemintedstyle{emacs}
\numberwithin{equation}{section}
\renewcommand{\arraystretch}{1.2} % space between table rows

% Global style setup
\makeatletter % change font size for not having underfull hbox
\renewcommand\Huge{\@setfontsize\Huge{22pt}{18}}
\makeatother

\AtBeginDocument{\renewcommand*\contentsname{Contents}}
\AtBeginDocument{\renewcommand*\refname{References}}
\setlength{\headheight}{40pt}
\pagestyle{fancy}
\fancyhead{} % clear all header fields
\fancyhead[L]{
  \begin{tabular}{rl}
    \begin{picture}(25,15)(0,0)
    \put(0,-8){\includegraphics[width=8mm, height=8mm]{./assets/hcmut.png}}
    \end{picture}
    \begin{tabular}{l}
      \textbf{\bf \ttfamily University of Technology, Ho Chi Minh City}\\
      \textbf{\bf \ttfamily Faculty of Computer Science and Engineering}
    \end{tabular}
  \end{tabular}
}
\fancyhead[R]{
	\begin{tabular}{l}
		\tiny \bf \\
		\tiny \bf
	\end{tabular}  }
\fancyfoot{} % clear all footer fields
\fancyfoot[L]{\scriptsize \ttfamily Programming Integration Project --- Academic year 2021--2022}
\fancyfoot[R]{\scriptsize \ttfamily Page {\thepage}/\pageref{LastPage}}
\renewcommand{\headrulewidth}{0.3pt}
\renewcommand{\footrulewidth}{0.3pt}

% \everymath{\color{blue}}

\newcommand*\mean[1]{\bar{#1}}
\newenvironment{code}[1]
{\VerbatimEnvironment%
  \begin{mdframed}[leftline=false,rightline=false,backgroundcolor=magenta!10,nobreak=false]%
    \begin{minted}[linenos=true,breaklines,breaksymbolleft=,obeytabs=true,tabsize=2]{#1}%
}
{
    \end{minted}%
  \end{mdframed}%
}

\begin{document}

\begin{titlepage}
  \begin{center}
    VIETNAM NATIONAL UNIVERSITY, HO CHI MINH CITY \\
    UNIVERSITY OF TECHNOLOGY \\
    FACULTY OF COMPUTER SCIENCE AND ENGINEERING
  \end{center}

  \vspace{1cm}

  \begin{figure}[H]
    \centering
    \includegraphics[width=0.5\textwidth]{./assets/hcmut.png}
  \end{figure}

  \vspace{1cm}

  \begin{center}
    \begin{tabular}{c}
      \textbf{\Large Programming Integration Project (CO3103)} \\
      {}                                                       \\
      \midrule                                                 \\
      \textbf{\Large Project Report: Phase 1}                  \\
      {}                                                       \\
      \textbf{\Huge Online Shopping Website}                   \\   % Sales website                            \\
      {}                                                       \\
      \bottomrule
    \end{tabular}
  \end{center}

  \vspace{3cm}

  \begin{table}[h]
    \begin{tabular}{rl}
      \hspace{1cm} Advisor: & Prof.\ Quản Thành Thơ \\
    \end{tabular}
  \end{table}

  \begin{center}
    {\footnotesize HO CHI MINH CITY, OCTOBER 2021}
  \end{center}
\end{titlepage}

\tableofcontents
\newpage

%\thispagestyle{empty}
\section*{Member list}
\begin{center}
  \begin{tabular}{llc}
    \toprule
    \textbf{No.} & \textbf{Full name} & \textbf{Student ID} \\
    \midrule
    1            & Nguyễn Hoàng       & 1952255             \\
    2            & Nguyễn Chính Khôi  & 1952793             \\
    3            & Vũ Anh Nhi         & 1952380             \\
    4            & Lương Duy Hưng     & 1952747             \\
    \bottomrule
  \end{tabular}
\end{center}

\newpage

\section{Topic}
This project focuses on creating a web-based application that allows users or customers to browse and make transactions.
The specific genre of product is phone, primarily smart mobile device. \\

This e-commerce application is analogous to real life phone selling retailers that require a more convenient and efficient way for customers to interact with their products.
Because of that, the requirements and features will mainly focus on practical demands as if it was proposed to solve the problem.

\section{Proposed Features}
\begin{center}
  \begin{tabular}{*{2}{l}}
    \toprule
    Name                             & Descriptions                                      \\
    \midrule
    Get news \& offers               & Display current news and offers                   \\
    View catalog                     & Display list of products                          \\
    Place orders                     & Customers can add items to their cart             \\
    Member account authentication    & Login with username and password, (optional: SSO) \\
    Member account balance \& points & Save  purchases to calculate points               \\
    Create coupons                   & Validate coupons when checking out                \\
    Display statistics               & Monthly sales of reports (Optional)               \\
    \bottomrule
  \end{tabular}
\end{center}

\section{Technologies}

% Advantages/ Disadvantages
% Reasons why we choose this
% Basic features

We decided to make the backend serve a RESTful API, so that we can decouple the server and client as well as ensuring ease of scale should that ever happen.
With that said, we chose to demonstrate this project as a web application.

\begin{enumerate}[label=\alph*.]

  \item Frontend: \textbf{React}

        React (React.js or ReactJS) is a free and open-source front-end JavaScript library for building user interfaces or UI components.
        Developed at Facebook and released in 2013, it can be said that React has been the most influential UI library of recent years.

        We use React to build components that represent logical reusable parts of the UI\@.
        The beauty of React is that the simplicity of building a component has been brought down to its theoretical minimum: a Javascript function.
        The return value from these functions is the HTML or UI, which is written in a special syntax called JSX, allowing easy combination of Javascript with Html markup.

        The main reason we want to use React is not the library itself but the massive ecosystem surrounding it.
        React itself does not care about routing state management, animation or anything like that.
        Instead, it lets those concerns evolve naturally within the open-source community.
        No matter what we are trying to do, there is a good chance that a good supporting library to help us get it done has already existed.

  \item Backend: \textbf{NestJS}

        NestJS is a Node.js framework for building scalable server-side applications with Typescript.
        It comes with a ton of built-in modules to work with databases, handle security, implement streaming and anything else you can imagine doing in a server-side application.

        Using the NestJS CLI, you can scaffold out a new project with a code base pre-configured with Jest for testing and set up with Typescript to write more readable and reliable code.

  \item Database: \textbf{PostgreSQL}

        PostgreSQL is a powerful, open source object-relational database system that uses and extends the SQL language combined with many features that safely store and scale the most complicated data workloads.

        With more than 30 years of active development on the core platform, PostgreSQL has earned a strong reputation for its proven architecture, reliability, data integrity and more.
\end{enumerate}

\end{document}
